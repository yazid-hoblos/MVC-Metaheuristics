\documentclass[12pt,a4paper]{article}
\usepackage[latin1]{inputenc}
\usepackage[T1]{fontenc}
\usepackage{amsmath}
\usepackage{listings}
\usepackage{xcolor}
\usepackage{geometry}
\geometry{margin=1in}

\lstset{
    language=Python,
    basicstyle=\ttfamily\footnotesize,
    breaklines=true,
    stringstyle=\color{gray},
    commentstyle=\color{gray},
    keywordstyle=\color{blue},
    showstringspaces=false,
    frame=single,
    captionpos=b,
    numbers=left,
    numberstyle=\tiny
}

\title{Appendix: Minimum Vertex Cover Optimization\\
Pseudocode and Implementation Details}
\date{\today}

\begin{document}

\maketitle

\section{Algorithm Pseudocode}

\subsection{Genetic Algorithm}

\begin{algorithm}[H]
\caption{Genetic Algorithm for Minimum Vertex Cover}
\begin{algorithmic}[1]
\Procedure{GeneticAlgorithm}{problem, encoding, fitness, params}
    \State population $\leftarrow$ InitializePopulation(params.population\_size, encoding)
    \State best\_solution $\leftarrow$ None
    \State best\_fitness $\leftarrow$ $-\infty$
    
    \For{generation $\leftarrow$ 1 to params.generations}
        \State fitness\_scores $\leftarrow$ EvaluatePopulation(population, fitness)
        
        \State gen\_best $\leftarrow$ ArgMax(fitness\_scores)
        \If{fitness\_scores[gen\_best] > best\_fitness}
            \State best\_fitness $\leftarrow$ fitness\_scores[gen\_best]
            \State best\_solution $\leftarrow$ DeepCopy(population[gen\_best])
        \EndIf
        
        \State elites, elite\_fitness $\leftarrow$ SelectElites(population, fitness\_scores, params.elitism\_rate)
        \State new\_population $\leftarrow$ elites
        
        \While{Length(new\_population) < params.population\_size}
            \State parent1 $\leftarrow$ Selection(population, fitness\_scores, params.selection\_type)
            \State parent2 $\leftarrow$ Selection(population, fitness\_scores, params.selection\_type)
            
            \If{Random() < params.crossover\_rate}
                \State child1, child2 $\leftarrow$ Crossover(parent1, parent2, encoding)
            \Else
                \State child1 $\leftarrow$ DeepCopy(parent1)
                \State child2 $\leftarrow$ DeepCopy(parent2)
            \EndIf
            
            \State Mutate(child1, params.mutation\_rate, encoding)
            \State Mutate(child2, params.mutation\_rate, encoding)
            
            \State new\_population.Append(child1)
            \If{Length(new\_population) < params.population\_size}
                \State new\_population.Append(child2)
            \EndIf
        \EndWhile
        
        \State population $\leftarrow$ new\_population[0:params.population\_size]
    \EndFor
    
    \State cover $\leftarrow$ encoding.SolutionToCover(best\_solution)
    \Return \{best\_solution, best\_fitness, cover, is\_valid\}
\EndProcedure
\end{algorithmic}
\end{algorithm}

\subsection{Simulated Annealing}

\begin{algorithm}[H]
\caption{Simulated Annealing for Minimum Vertex Cover}
\begin{algorithmic}[1]
\Procedure{SimulatedAnnealing}{problem, encoding, fitness, params}
    \State current $\leftarrow$ RandomSolution(encoding)
    \State current\_fitness $\leftarrow$ Evaluate(current, fitness, encoding)
    \State best $\leftarrow$ DeepCopy(current)
    \State best\_fitness $\leftarrow$ current\_fitness
    
    \State temperature $\leftarrow$ params.initial\_temperature
    \State iteration $\leftarrow$ 0
    
    \While{temperature > params.min\_temperature AND iteration < params.max\_iterations}
        \For{i $\leftarrow$ 1 to params.iterations\_per\_temperature}
            \State neighbor $\leftarrow$ GetNeighbor(current, encoding)
            \State neighbor\_fitness $\leftarrow$ Evaluate(neighbor, fitness, encoding)
            
            \State accept\_prob $\leftarrow$ AcceptanceProbability(current\_fitness, neighbor\_fitness, temperature)
            
            \If{Random() < accept\_prob}
                \State current $\leftarrow$ neighbor
                \State current\_fitness $\leftarrow$ neighbor\_fitness
                
                \If{current\_fitness > best\_fitness}
                    \State best $\leftarrow$ DeepCopy(current)
                    \State best\_fitness $\leftarrow$ current\_fitness
                \EndIf
            \EndIf
            
            \State iteration $\leftarrow$ iteration + 1
            \If{iteration $\geq$ params.max\_iterations}
                \State \textbf{break}
            \EndIf
        \EndFor
        
        \State temperature $\leftarrow$ temperature $\times$ params.cooling\_rate
    \EndWhile
    
    \State cover $\leftarrow$ encoding.SolutionToCover(best)
    \Return \{best, best\_fitness, cover, is\_valid\}
\EndProcedure

\Function{AcceptanceProbability}{current\_fitness, neighbor\_fitness, temperature}
    \If{neighbor\_fitness $\geq$ current\_fitness}
        \Return 1.0
    \Else
        \Return $\exp\left(\frac{\text{neighbor\_fitness} - \text{current\_fitness}}{\text{temperature}}\right)$
    \EndIf
\EndFunction
\end{algorithmic}
\end{algorithm}

\subsection{Tabu Search}

\begin{algorithm}[H]
\caption{Tabu Search for Minimum Vertex Cover}
\begin{algorithmic}[1]
\Procedure{TabuSearch}{problem, encoding, fitness, params}
    \State current $\leftarrow$ RandomSolution(encoding)
    \State current\_fitness $\leftarrow$ Evaluate(current, fitness, encoding)
    \State best $\leftarrow$ DeepCopy(current)
    \State best\_fitness $\leftarrow$ current\_fitness
    
    \State tabu\_list $\leftarrow$ EmptyQueue(max\_size=params.tabu\_list\_size)
    \State tabu\_list.Add(Hash(current))
    
    \State iteration $\leftarrow$ 0
    \State no\_improvement\_count $\leftarrow$ 0
    
    \While{iteration < params.max\_iterations AND no\_improvement\_count < 100}
        \State neighbors $\leftarrow$ GenerateNeighborhood(current, encoding)
        
        \If{neighbors is empty}
            \State \textbf{break}
        \EndIf
        
        \State best\_neighbor $\leftarrow$ None
        \State best\_neighbor\_fitness $\leftarrow$ $-\infty$
        
        \For{\textbf{each} neighbor \textbf{in} neighbors}
            \State neighbor\_hash $\leftarrow$ Hash(neighbor)
            \State is\_tabu $\leftarrow$ tabu\_list.Contains(neighbor\_hash)
            \State neighbor\_fitness $\leftarrow$ Evaluate(neighbor, fitness, encoding)
            
            \State aspiration\_met $\leftarrow$ (params.aspiration AND neighbor\_fitness > best\_fitness)
            
            \If{(NOT is\_tabu AND neighbor\_fitness > best\_neighbor\_fitness) OR}
            \phantom{\If{(}aspiration\_met}
                \State best\_neighbor $\leftarrow$ neighbor
                \State best\_neighbor\_fitness $\leftarrow$ neighbor\_fitness
            \EndIf
        \EndFor
        
        \If{best\_neighbor is None}
            \State \textbf{break}
        \EndIf
        
        \State current $\leftarrow$ best\_neighbor
        \State current\_fitness $\leftarrow$ best\_neighbor\_fitness
        \State tabu\_list.Add(Hash(current))
        
        \If{current\_fitness > best\_fitness}
            \State best $\leftarrow$ DeepCopy(current)
            \State best\_fitness $\leftarrow$ current\_fitness
            \State no\_improvement\_count $\leftarrow$ 0
        \Else
            \State no\_improvement\_count $\leftarrow$ no\_improvement\_count + 1
        \EndIf
        
        \State iteration $\leftarrow$ iteration + 1
    \EndWhile
    
    \State cover $\leftarrow$ encoding.SolutionToCover(best)
    \Return \{best, best\_fitness, cover, is\_valid\}
\EndProcedure
\end{algorithmic}
\end{algorithm}

\section{Encoding Implementation}

\subsection{Binary Encoding Example}

\begin{lstlisting}[caption=Binary Encoding - Solution to Cover Conversion]
def binary_solution_to_cover(solution, num_nodes):
    """Convert binary vector to vertex cover."""
    cover = set()
    for i in range(num_nodes):
        if solution[i] == 1:
            cover.add(i)
    return cover

def binary_cover_to_solution(cover, num_nodes):
    """Convert cover to binary vector."""
    solution = [0] * num_nodes
    for node in cover:
        solution[node] = 1
    return solution

def binary_random_solution(num_nodes):
    """Generate random binary vector."""
    return [random.randint(0, 1) for _ in range(num_nodes)]

def binary_mutation(solution, mutation_rate, num_nodes):
    """Bit-flip mutation."""
    for i in range(num_nodes):
        if random.random() < mutation_rate:
            solution[i] = 1 - solution[i]

def binary_crossover(parent1, parent2):
    """Uniform crossover for binary vectors."""
    child1, child2 = [], []
    for i in range(len(parent1)):
        if random.random() < 0.5:
            child1.append(parent1[i])
            child2.append(parent2[i])
        else:
            child1.append(parent2[i])
            child2.append(parent1[i])
    return child1, child2
\end{lstlisting}

\subsection{Set-Based Encoding Example}

\begin{lstlisting}[caption=Set-Based Encoding - Solution to Cover Conversion]
def set_solution_to_cover(solution):
    """Convert sorted list to cover set."""
    return set(solution)

def set_cover_to_solution(cover):
    """Convert cover to sorted list."""
    return sorted(list(cover))

def set_random_solution(num_nodes):
    """Generate random set of nodes."""
    num_selected = random.randint(1, max(2, num_nodes // 2))
    nodes = random.sample(range(num_nodes), num_selected)
    return sorted(nodes)

def set_mutation(solution, mutation_rate, num_nodes):
    """Add/remove nodes from set."""
    if random.random() < mutation_rate / 2 and len(solution) > 0:
        # Remove random node
        idx = random.randint(0, len(solution) - 1)
        solution.pop(idx)
    
    if random.random() < mutation_rate / 2:
        # Add random node not in cover
        node = random.randint(0, num_nodes - 1)
        if node not in solution:
            solution.append(node)
            solution.sort()

def set_crossover(parent1, parent2):
    """Single-point crossover for variable-length sets."""
    # For variable-length, perform standard single-point split
    # and recombine (alternative: intersection/union-based)
    if len(parent1) <= 1:
        return list(parent1), list(parent2)
    
    point = random.randint(1, len(parent1) - 1)
    child1 = parent1[:point] + parent2[point:]
    child2 = parent2[:point] + parent1[point:]
    return child1, child2
\end{lstlisting}

\section{Fitness Function Implementation}

\begin{lstlisting}[caption=Fitness Function Examples]
def cover_size_minimization(cover, problem):
    """Fitness = 1 / (1 + normalized_size)"""
    size = len(cover)
    normalized_size = size / problem.num_nodes
    return 1.0 / (1.0 + normalized_size)

def constraint_penalty(cover, problem, penalty_weight=1.0):
    """
    Fitness with constraint penalty.
    Valid covers get positive fitness; invalid get penalized.
    """
    is_valid = problem.is_valid_cover(cover)
    uncovered = count_uncovered_edges(cover, problem)
    cover_size = len(cover)
    
    if is_valid:
        return 1.0 - (cover_size / problem.num_nodes) * 0.5
    else:
        penalty = penalty_weight * (uncovered + cover_size / problem.num_nodes)
        return max(0.0, 1.0 - penalty)

def edge_coverage_optimization(cover, problem, size_weight=0.3):
    """
    Multi-objective: maximize edge coverage, minimize cover size.
    Fitness = (covered_edges / total) - weight * (size / nodes)
    """
    covered = count_covered_edges(cover, problem)
    coverage_ratio = covered / problem.num_edges if problem.num_edges > 0 else 0
    
    size_penalty = (len(cover) / problem.num_nodes) * size_weight
    
    fitness = coverage_ratio - size_penalty
    return max(0.0, fitness)

def count_uncovered_edges(cover, problem):
    """Count edges not covered by solution."""
    uncovered = 0
    for u, v in problem.edges:
        if u not in cover and v not in cover:
            uncovered += 1
    return uncovered

def count_covered_edges(cover, problem):
    """Count edges covered by solution."""
    covered = 0
    for u, v in problem.edges:
        if u in cover or v in cover:
            covered += 1
    return covered
\end{lstlisting}

\section{Experimental Framework}

\begin{lstlisting}[caption=Main Experiment Loop Structure]
def run_experiments(num_runs=5):
    """Run full experimental suite."""
    instances = generate_benchmark_instances()
    results = []
    
    for problem, instance_name in instances:
        print(f"Processing {instance_name}...")
        
        encodings = [
            BinaryEncoding(),
            SetEncoding(),
            EdgeCentricEncoding(problem.edges)
        ]
        
        fitness_functions = [
            CoverSizeMinimization(problem),
            ConstraintPenalty(problem, penalty_weight=1.0),
            EdgeCoverageOptimization(problem, size_weight=0.3)
        ]
        
        for run_id in range(num_runs):
            for encoding in encodings:
                for fitness_func in fitness_functions:
                    # Genetic Algorithm
                    ga = GeneticAlgorithm(
                        problem, encoding, fitness_func,
                        params=GAParams(population_size=100, generations=300)
                    )
                    result_ga = ga.run()
                    results.append({
                        'instance': instance_name,
                        'algorithm': 'GA',
                        'encoding': encoding.get_name(),
                        'fitness_func': fitness_func.get_name(),
                        'run': run_id,
                        'cover_size': result_ga['best_cover_size'],
                        'is_valid': result_ga['is_valid'],
                        'fitness': result_ga['best_fitness']
                    })
                    
                    # Simulated Annealing
                    sa = SimulatedAnnealing(
                        problem, encoding, fitness_func,
                        params=SAParams(initial_temperature=100.0, max_iterations=5000)
                    )
                    result_sa = sa.run()
                    results.append({...})  # Similar structure
                    
                    # Tabu Search
                    ts = TabuSearch(
                        problem, encoding, fitness_func,
                        params=TSParams(tabu_list_size=50, max_iterations=5000)
                    )
                    result_ts = ts.run()
                    results.append({...})  # Similar structure
    
    # Save and analyze results
    save_to_csv(results, 'results.csv')
    print_summary_statistics(results)
    return results
\end{lstlisting}

\section{Instance Generation}

\begin{lstlisting}[caption=Benchmark Instance Generation]
import networkx as nx
import random

def generate_erdos_renyi_instance(num_nodes, edge_probability, seed=None):
    """Generate random graph using ER model."""
    if seed:
        random.seed(seed)
    
    graph = nx.erdos_renyi_graph(num_nodes, edge_probability)
    return MinimumVertexCoverProblem(graph)

def generate_scale_free_instance(num_nodes, m=2, seed=None):
    """Generate scale-free graph (Barabasi-Albert model)."""
    if seed:
        random.seed(seed)
    
    graph = nx.barabasi_albert_graph(num_nodes, m)
    return MinimumVertexCoverProblem(graph)

def generate_benchmark_suite():
    """Generate standard benchmark instances."""
    instances = []
    
    # Small ER instance
    instances.append((
        generate_erdos_renyi_instance(20, 0.3, seed=42),
        "small_20nodes"
    ))
    
    # Medium ER instance
    instances.append((
        generate_erdos_renyi_instance(50, 0.25, seed=43),
        "medium_50nodes"
    ))
    
    # Large ER instance
    instances.append((
        generate_erdos_renyi_instance(100, 0.15, seed=44),
        "large_100nodes"
    ))
    
    # Scale-free instance
    instances.append((
        generate_scale_free_instance(50, m=3, seed=45),
        "scale_free_50nodes"
    ))
    
    return instances
\end{lstlisting}

\section{Key Implementation Details}

\subsection{Parameter Selection Rationale}

\begin{enumerate}
    \item \textbf{GA Population Size (100)}: Balances diversity against computational cost. Too small (20-30) causes premature convergence; too large (500+) slows iterations.
    
    \item \textbf{GA Generations (300)}: Approximately 30,000 fitness evaluations. Provides reasonable runtime while avoiding convergence plateau.
    
    \item \textbf{SA Initial Temperature (100.0)}: Calibrated so that $\exp(-100/100) = \exp(-1) \approx 0.37$, giving ~37\% acceptance of random moves initially.
    
    \item \textbf{SA Cooling Rate (0.95)}: Geometric schedule slower than 0.9 (too aggressive) but faster than 0.99 (too slow). Reaches min temperature in ~90 iterations.
    
    \item \textbf{TS Tabu List Size (50)}: Approximately 10\% of solution space (for 20-100 node graphs). Prevents recent cycling without over-constraining.
\end{enumerate}

\subsection{Reproducibility}

All experiments use fixed random seeds:
\begin{itemize}
    \item Instance generation: seeds 42-45
    \item Algorithm runs: Python random.seed() and numpy.random.seed() set before each run
    \item Enables exact replication of reported results
\end{itemize}

\end{document}
